\documentclass[a4paper,titlepage,halfparskip,12pt]{scrreprt}

\usepackage[ngerman]{babel, varioref}
\usepackage[utf8]{inputenc}
\usepackage[T1]{fontenc}
\usepackage{graphicx}
\usepackage{fancyhdr}
\usepackage{amsmath}
\usepackage{geometry}
\geometry{a4paper, top=25mm,left=25mm,right=25mm,bottom=25mm, footskip=12mm}
\usepackage{longtable}
\usepackage{setspace}
\usepackage{lmodern}
%blocksatz
\sloppy
%formatierung literaturverzeichnisangabe
\bibliographystyle{unsrt}

%Auflistungen von Punkten
\usepackage{paralist} 
%urls anzeigen
\usepackage{url}

%Codelisting
\usepackage{xcolor}
\definecolor{mygreen}{rgb}{0,0.6,0}
\definecolor{mygray}{rgb}{0.5,0.5,0.5}
\definecolor{mymauve}{rgb}{0.58,0,0.82}
\definecolor{burntorange}{rgb}{0.8, 0.33, 0.0}
\definecolor{cornellred}{rgb}{0.7, 0.11, 0.11}

\usepackage{listingsutf8}
\lstset{
commentstyle=\color{mygreen},
numberstyle=\small\color{black},
stringstyle=\color{mymauve},
emph={square}, 
showstringspaces=false,
flexiblecolumns=false,
tabsize=2,
numbers=left,
numberblanklines=false,
stepnumber=1,
captionpos=b,
numbersep=5pt,
xleftmargin=15pt,
breaklines=true,
inputencoding=utf8,
extendedchars=true,
extendedchars=true,
basicstyle=\ttfamily\footnotesize,
keywordstyle = \bfseries\color{burntorange},
keywordstyle = [2]\bfseries\color{cornellred},
literate=%
    {Ä}{{\"A}}1%
    {Ö}{{\"O}}1%
    {Ü}{{\"U}}1%
    {ä}{{\"a}}1%
    {ö}{{\"o}}1%
    {ü}{{\"u}}1%
    {ß}{{\ss}}1,%
frame=single,
frameround=ffff
}

%meta data
\usepackage[hidelinks]{hyperref}
\urlstyle{same}

%akronymverzeichnis
\usepackage[printonlyused]{acronym}

% titel definieren
\newcommand{\titel}{Entwicklung eines Chatsystems\\auf Basis von XMPP}

%autor definieren
\newcommand{\autor}{Lukas Priester,Oliver Klapper}
\newcommand{\keywords}{\autor,\titel,Studienarbeit}

% Allgemeines für das PDF
\hypersetup{
    pdftitle={\titel},
    pdfauthor={\autor},
    pdfcreator={\autor},
    pdfsubject={\titel},
    pdflang={Deutsch},
    pdfdisplaydoctitle=true,
    pdfkeywords={\keywords},
}

% set distances of chapter headlines in document
\renewcommand*\chapterheadstartvskip{\vspace*{20pt}} % set distance to header
% set distance to text
%\renewcommand*\chapterheadendvskip{%
%  \vspace*{1\baselineskip plus .1\baselineskip minus .167\baselineskip}}


\begin{document}

\begin{table}[h]
\centering
\begin{tabular}{lcr}
\includegraphics[height=3.5cm]{images/dhbw-logo}
\end{tabular}
\end{table}
\bigskip
\bigskip
\begin{center}
\vspace*{12mm} {\LARGE\textbf{\titel}}\\
\vspace*{12mm} {\large\textbf{Studienarbeit}}\\
\vspace*{3mm} {\large\textbf{5. - 6. Semester}}\\
\vspace*{12mm} des Studiengangs Informationstechnik (B.Sc.)\\ an der Dualen Hochschule Baden-Württemberg Stuttgart\\
% \vspace*{3mm} an der Dualen Hochschule Baden-Württemberg\\
\vspace*{12mm} von\\
\vspace*{3mm} {\large\textbf{Lukas Priester, Oliver Klapper}}\\
\vspace*{12mm} \today\\
\end{center}
\vfill
\begin{spacing}{1.5}
\begin{tabbing}
mmmmmmmmmmmmmmmmmmmmmmmmmm \= \kill
\textbf{Bearbeitungszeitraum} \> 01.10.2019 - 01.05.2020\\
\textbf{Matrikelnummer, Kurs} \> 7288057, 4191693 \\
\textbf{Kurs} \> TINF17IN\\
\textbf{Betreuer der Hochschule} \> Alfred Becker\\
\textbf{Gutachter der Hochschule} \> Alfred Becker\\
\end{tabbing}
\end{spacing}
%Seitennummerierung ausschalten
\pagenumbering{gobble}
\newpage

\section*{Selbstständigkeitserklärung}

\bigskip

Ich versichere hiermit, dass ich meine Bachelorarbeit (bzw. Studien- und Projektarbeit) mit dem Thema:

\smallskip

%% eigentlich hier \titel verwenden statt duplicated titel, aber umbruch erzwingt
%% doppeltes ausschreiben des titels
\texttt{Entwicklung eines Chatsystems auf Basis von XMPP}

\smallskip

selbstständig verfasst und keine anderen als die angegebenen Quellen und Hilfsmittel benutzt habe.

\bigskip

Ich versichere zudem, dass die eingereichte elektronische Fassung mit der gedruckten Fassung übereinstimmt.*

\bigskip

\begin{small}

* falls beide Fassungen gefordert sind

\bigskip

\bigskip

\noindent\begin{tabular}{ll}
\makebox[2.5in]{\hrulefill} & \makebox[2.5in]{\hrulefill}\\
Ort, Datum & Unterschrift
\end{tabular}
\end{small}

\newpage

%abstract text
\section*{Abstract}

\newpage

%inhaltsverzeichnis
	% Inhaltsverzeichnis
	\cleardoublepage
	\begin{spacing}{1.1}
		\begingroup
		
			% auskommentieren für Seitenzahlen unter Inhaltsverzeichnis
			\renewcommand*{\chapterpagestyle}{empty}
			\pagestyle{empty}
			
			
			%\setcounter{tocdepth}{1}
			%für die Anzeige von Unterkapiteln im Inhaltsverzeichnis
			\setcounter{tocdepth}{2}
			
			\tableofcontents
			\clearpage
		\endgroup
	\end{spacing}

%% new header/footer settings
\renewcommand{\sectionmark}[1]{\markright{\thesection\ #1}} % make header rightmark
\fancypagestyle{fancyheadlines}{
\pagenumbering{arabic}
\fancyhf{}
\lhead{\slshape\rightmark}
%%\rhead{\slshape\nouppercase{\leftmark}}
\renewcommand{\headrulewidth}{0.4pt}
%\lfoot{\slshape DHBW Stuttgart | Lukas Priester, Oliver Klapper}
\cfoot{\thepage}
\renewcommand{\footrulewidth}{0.4pt}
}

% Redefine the plain page style, show only page number in figure,table,...contents
% and chapter pages
\fancypagestyle{plain}{%
  \fancyhf{}%
  %\lfoot{\slshape DHBW Stuttgart | Lukas Priester, Oliver Klapper}%
  \cfoot{\thepage}
  \renewcommand{\headrulewidth}{0pt}% Line at the header invisible
  \renewcommand{\footrulewidth}{0.4pt}% Line at the footer visible
}


\newpage
\pagenumbering{Roman}


%abkürzungsverzeichnis
\cleardoublepage
\addcontentsline{toc}{chapter}{Abkürzungsverzeichnis}
\chapter*{Abkürzungsverzeichnis}
\begin{acronym}[YTMMM]
\setlength{\itemsep}{-\parsep}

\acro{HTML}{HyperText Markup Language}
\end{acronym}

%abbildungsverzeichnis
\cleardoublepage
\addcontentsline{toc}{chapter}{\listfigurename}
\listoffigures
\newpage
%tabellenverzeichnis
\cleardoublepage
\addcontentsline{toc}{chapter}{\listtablename}
\listoftables
\newpage
%listingverzeichnis
\cleardoublepage
\addcontentsline{toc}{chapter}{\lstlistingname}
\lstlistoflistings
\newpage

\begin{onehalfspacing}

%% header and footer settings
\pagestyle{fancyheadlines}

\chapter{Einleitung}
\label{chap:Einleitung}


\section{Aufgabenstellung}
\label{sec:Aufgabenstellung}


\section{Ziele der Arbeit}
\label{sec:Ziele}


\section{Stand der Technik}
\label{sec:StandDerTechnik}

\newpage

\chapter{Datenschutzrechtliche Aspekte}
\label{chap:Datenschutz}

\newpage

\chapter{Anforderungen}
\label{Anforderungen}

\newpage

\chapter{Theoretische Grundlagen}
\label{chap:Theorie}

\section{XMPP-Protokoll}
\label{sec:XMPP}

\section{Python}
\label{sec:Python}

\section{Chatserver}
\label{sec:Chatserver}

\begin{lstlisting}[language=Python,caption=Example Listing Python,label={lst:Example}]
"""The first step is to create an SMTP object, each object is used for connection 
with one server."""

import smtplib
server = smtplib.SMTP('smtp.gmail.com', 587)

#Next, log in to the server
server.login("youremailusername", "password")

#Send the mail
msg = "
Hello!" # The /n separates the message from the headers
server.sendmail("you@gmail.com", "target@example.com", msg)
\end{lstlisting}


\section{Datenbank}
\label{sec:Datenbank}

\newpage

\chapter{Umsetzung und Implementierung}
\label{chap:Umsetzung}

\section{Vorbereitung}
\label{sec:Vorbereitung}

\section{Implementierung des Chatservers ejabberd}
\label{sec:ChatserverEntwicklung}

\subsection{Anbindung des Chatservers}
\label{subsec:Anbindung}

\subsection{Security}
\label{subsec:Security}

\subsection{Konfiguration von ejabberd}
\label{subsec:Konfiguration}

\newpage

\section{Entwicklung des Chat-Clients}
\label{sec:ClientEntwicklung}

\ac{HTML}

\subsection{Entwicklung der GUI}
\label{subsec:EntwicklungGUI}

\subsection{Implemintierung des Backends}
\label{subsec:Backend}

\newpage

\chapter{Datenanalyse}
\label{chap:Datenanalyse}

\newpage

\chapter{Ergebnis}
\label{chap:Ergebnis}

\newpage

\chapter{Fazit}
\label{chap:Fazit}


\end{onehalfspacing}
\newpage

%% \bibliography{Literatur}
\newpage
%setze Anhang mit appendix
%addpart, sodass Anhang im Inhaltsverzeichnis ohne Buchstaben auftaucht
\appendix
\addpart{Anhang}

\end{document}