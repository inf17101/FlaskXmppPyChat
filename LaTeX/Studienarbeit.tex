\documentclass[a4paper,titlepage,halfparskip,12pt]{scrreprt}

\usepackage[ngerman]{babel, varioref}
\usepackage[utf8]{inputenc}
\usepackage[T1]{fontenc}
\usepackage{graphicx}
\usepackage{fancyhdr}
\usepackage{amsmath}
\usepackage{geometry}
\geometry{a4paper, top=25mm,left=25mm,right=25mm,bottom=25mm, footskip=12mm}
\usepackage{longtable}
\usepackage{setspace}
\usepackage{lmodern}
%blocksatz
\sloppy
%formatierung literaturverzeichnisangabe
\bibliographystyle{unsrt}

%Auflistungen von Punkten
\usepackage{paralist} 
%urls anzeigen
\usepackage{url}

%Codelisting
\usepackage{xcolor}
\definecolor{mygreen}{rgb}{0,0.6,0}
\definecolor{mygray}{rgb}{0.5,0.5,0.5}
\definecolor{mymauve}{rgb}{0.58,0,0.82}
\definecolor{burntorange}{rgb}{0.8, 0.33, 0.0}
\definecolor{cornellred}{rgb}{0.7, 0.11, 0.11}

\usepackage{listingsutf8}
\lstset{
commentstyle=\color{mygreen},
numberstyle=\small\color{black},
stringstyle=\color{mymauve},
emph={square}, 
showstringspaces=false,
flexiblecolumns=false,
tabsize=2,
numbers=left,
numberblanklines=false,
stepnumber=1,
captionpos=b,
numbersep=5pt,
xleftmargin=15pt,
breaklines=true,
inputencoding=utf8,
extendedchars=true,
extendedchars=true,
basicstyle=\ttfamily\footnotesize,
keywordstyle = \bfseries\color{burntorange},
keywordstyle = [2]\bfseries\color{cornellred},
literate=%
    {Ä}{{\"A}}1%
    {Ö}{{\"O}}1%
    {Ü}{{\"U}}1%
    {ä}{{\"a}}1%
    {ö}{{\"o}}1%
    {ü}{{\"u}}1%
    {ß}{{\ss}}1,%
frame=single,
frameround=ffff
}

%meta data
\usepackage[hidelinks]{hyperref}
\urlstyle{same}

%akronymverzeichnis
\usepackage[printonlyused]{acronym}

% titel definieren
\newcommand{\titel}{Entwicklung eines Chatsystems\\auf Basis von XMPP}

%autor definieren
\newcommand{\autor}{Lukas Priester,Oliver Klapper}
\newcommand{\keywords}{\autor,\titel,Studienarbeit}

% Allgemeines für das PDF
\hypersetup{
    pdftitle={\titel},
    pdfauthor={\autor},
    pdfcreator={\autor},
    pdfsubject={\titel},
    pdflang={Deutsch},
    pdfdisplaydoctitle=true,
    pdfkeywords={\keywords},
}

% set distances of chapter headlines in document
\renewcommand*\chapterheadstartvskip{\vspace*{20pt}} % set distance to header
% set distance to text
%\renewcommand*\chapterheadendvskip{%
%  \vspace*{1\baselineskip plus .1\baselineskip minus .167\baselineskip}}


\begin{document}

\begin{table}[h]
\centering
\begin{tabular}{lcr}
\includegraphics[height=3.5cm]{images/dhbw-logo}
\end{tabular}
\end{table}
\bigskip
\bigskip
\begin{center}
\vspace*{12mm} {\LARGE\textbf{\titel}}\\
\vspace*{12mm} {\large\textbf{Studienarbeit}}\\
\vspace*{3mm} {\large\textbf{5. - 6. Semester}}\\
\vspace*{12mm} des Studiengangs Informationstechnik (B.Sc.)\\ an der Dualen Hochschule Baden-Württemberg Stuttgart\\
% \vspace*{3mm} an der Dualen Hochschule Baden-Württemberg\\
\vspace*{12mm} von\\
\vspace*{3mm} {\large\textbf{Lukas Priester, Oliver Klapper}}\\
\vspace*{12mm} \today\\
\end{center}
\vfill
\begin{spacing}{1.5}
\begin{tabbing}
mmmmmmmmmmmmmmmmmmmmmmmmmm \= \kill
\textbf{Bearbeitungszeitraum} \> 01.10.2019 - 01.05.2020\\
\textbf{Matrikelnummer, Kurs} \> 7288057, 4191693 \\
\textbf{Kurs} \> TINF17IN\\
\textbf{Betreuer der Hochschule} \> Alfred Becker\\
\textbf{Gutachter der Hochschule} \> Alfred Becker\\
\end{tabbing}
\end{spacing}
%Seitennummerierung ausschalten
\pagenumbering{gobble}
\newpage

\section*{Selbstständigkeitserklärung}

\bigskip

Ich versichere hiermit, dass ich meine Bachelorarbeit (bzw. Studien- und Projektarbeit) mit dem Thema:

\smallskip

%% eigentlich hier \titel verwenden statt duplicated titel, aber umbruch erzwingt
%% doppeltes ausschreiben des titels
\texttt{Entwicklung eines Chatsystems auf Basis von XMPP}

\smallskip

selbstständig verfasst und keine anderen als die angegebenen Quellen und Hilfsmittel benutzt habe.

\bigskip

Ich versichere zudem, dass die eingereichte elektronische Fassung mit der gedruckten Fassung übereinstimmt.*

\bigskip

\begin{small}

* falls beide Fassungen gefordert sind

\bigskip

\bigskip

\noindent\begin{tabular}{ll}
\makebox[2.5in]{\hrulefill} & \makebox[2.5in]{\hrulefill}\\
Ort, Datum & Unterschrift
\end{tabular}
\end{small}

\newpage

%abstract text
\section*{Abstract}

\newpage

%inhaltsverzeichnis
	% Inhaltsverzeichnis
	\cleardoublepage
	\begin{spacing}{1.1}
		\begingroup
		
			% auskommentieren für Seitenzahlen unter Inhaltsverzeichnis
			\renewcommand*{\chapterpagestyle}{empty}
			\pagestyle{empty}
			
			
			%\setcounter{tocdepth}{1}
			%für die Anzeige von Unterkapiteln im Inhaltsverzeichnis
			\setcounter{tocdepth}{2}
			
			\tableofcontents
			\clearpage
		\endgroup
	\end{spacing}

%% new header/footer settings
\renewcommand{\sectionmark}[1]{\markright{\thesection\ #1}} % make header rightmark
\fancypagestyle{fancyheadlines}{
\pagenumbering{arabic}
\fancyhf{}
\lhead{\slshape\rightmark}
%%\rhead{\slshape\nouppercase{\leftmark}}
\renewcommand{\headrulewidth}{0.4pt}
%\lfoot{\slshape DHBW Stuttgart | Lukas Priester, Oliver Klapper}
\cfoot{\thepage}
\renewcommand{\footrulewidth}{0.4pt}
}

% Redefine the plain page style, show only page number in figure,table,...contents
% and chapter pages
\fancypagestyle{plain}{%
  \fancyhf{}%
  %\lfoot{\slshape DHBW Stuttgart | Lukas Priester, Oliver Klapper}%
  \cfoot{\thepage}
  \renewcommand{\headrulewidth}{0pt}% Line at the header invisible
  \renewcommand{\footrulewidth}{0.4pt}% Line at the footer visible
}


\newpage
\pagenumbering{Roman}


%abkürzungsverzeichnis
\cleardoublepage
\addcontentsline{toc}{chapter}{Abkürzungsverzeichnis}
\chapter*{Abkürzungsverzeichnis}
\begin{acronym}[YTMMM]
\setlength{\itemsep}{-\parsep}

\acro{IMS} {Instant Messaging System}
\acro{XMPP} {Extensible Messaging and Presence Protocol}
\acro{XML} {Extensible Markup Language}
\acro{TCP} {Transmission Control Protocol}
\acro{TLS} {Transport Layer Security}
\acro{MUC} {Multi-User-Chat}
\acro{NLP} {Natural Language Processing}
\acro{MUC} {Multi User Chat}
\acro{ICQ} {I seek you}
\acro{VoIP} {Voice over IP}
\acro{HTML}{HyperText Markup Language}
\acro{DBMS}{Datenbankenmanagementsystem}
\acro{SQL}{Structured Query Language}
\acro{GUI}{graphical user interface}
\acro{WSGI}{Web Server Gateway Interface}
\acro{CSS}{Cascading Style Sheets}
\acro{ORM}{Object Relational Mapper}
\end{acronym}

%abbildungsverzeichnis
\cleardoublepage
\addcontentsline{toc}{chapter}{\listfigurename}
\listoffigures
\newpage
%tabellenverzeichnis
\cleardoublepage
\addcontentsline{toc}{chapter}{\listtablename}
\listoftables
\newpage
%listingverzeichnis
\cleardoublepage
\addcontentsline{toc}{chapter}{\lstlistingname}
\lstlistoflistings
\newpage

\begin{onehalfspacing}

%% header and footer settings
\pagestyle{fancyheadlines}

\chapter{Einleitung}
\label{chap:Einleitung}

Social Media (deutsche Übersetzung: \glqq soziale Medien\grqq{}) ist ein aktuelles und wichtiges gesellschaftliches Thema. Die vielfältigen und breitgefächerten Nutzungsmöglichkeiten beeinflussen das Privat- und das Berufsleben \cite{gabriel2017social}. Gabriel und Röhrs definieren Social Media in \cite{gabriel2017social} als die Verwendung digitaler Medien unter Einsatz computergeschützter Technologien, das heißt, von Hardware- und Softwaresystemen. Sie definieren den Nutzen von Social Media darin, dass Menschen Informationen suchen, erstellen, verteilen und austauschen können. Es gibt eine große Anzahl an unterschiedlichen Definitionen von Social Media. Nach Liu Yinyuan ist Social Media längst wichtiger Bestandteil des Unternehmensmarketings in Deutschland. In seinem Werk \glqq Social Media in China\grqq{} \cite{liu2016social} beschreibt er, dass in Unternehmen nicht mehr über die grundsätzliche Frage debattiert wird, ob Social Media für das Unternehmensmarketing eingesetzt werden soll, sondern wo und wie der Einsatz zielgerichtet erfolgen kann. Heutige Unternehmen sind gekennzeichnet von computergestützten Anwendungssystemen, die in allen Funktionsbereichen zur Planung, Steuerung und Kontrolle der Geschäftsprozesse und zu ihrer Verwaltung eingesetzt werden. Sie setzen dieses System in der B2B-Kommunikation (Business-To-Business-Kommunikation) ein. Immer wichtiger werden \textbf{\ac{IMS}}, wie zum Beispiel Whatsapp, Telegram, iMessage und Jabber, die dazu dienen intern Informationen schnell im Unternehmen zu verbreiten, hoch verfügbar zu machen und um Geschäftsprozesse standortunabhängig steuern zu können \cite{gabriel2017social}. Zusätzlich werden \textbf{Instant Messaging Systeme} auch zur externen Kommunikation benutzt. Nach Gabriel und Röhrs ist es möglich, dass mehrere Unternehmen mit Hilfe von innovativen Kommunikationstechniken zur Erreichung eines gemeinsamen Ziels besser kooperieren können. Außerdem ist die schnelle und direkte Kontaktaufnahme von Kunden über ein Messaging System zum Support eines Unternehmens eine einfache und schnelle Möglichkeit, Fragen zum Produkt ohne langes Warten in der Hotline zu stellen. Nach \cite{b2bmehner} werden Nachrichten von \textbf{Instant Message Systemen} im Gegensatz zu einer E-Mail in Echtzeit übertragen und dem Empfänger direkt zugestellt. Laut einer Statistik von \textbf{statista} benutzen 1,5 Milliarden Nutzer in Deutschland pro WhatsApp pro Monat \cite{statistaIMS}. Der Anteil der Nutzer von Whatsapp in Deutschland beträgt 75 Prozent an der Gesamtbevölkerung \cite{statistaIMS}. Die Statistik zeigt, dass \textbf{Instant Messaging Systeme} eine Möglichkeit in der Zukunft darstellen Kunden direkter anzusprechen oder Support zu gewährleisten. Zusätzlich zu Nachrichtendiensten werden künstliche Intelligenzen und Algorithmen benötigt, die Nutzer unterstützen oder gesammelte Daten von Benutzern charakterisieren oder interpretieren können. Algorithmen werden verwendet, um zum Beispiel die emotionale Befindlichkeit anhand eines Text einzuordnen, um Suizid-Gedanken frühzeitig zu erkennen \cite{stasytisIME}. Ein weiteres Anwendungsszenario sind Text-to-Speech (deutsche Übersetzung: Text-zu-Sprache) Funktionalitäten, bei denen gesprochene Worte des Nutzers durch Algorithmen in Text umgewandelt werden, sodass ein Nutzer die Nachricht nicht mehr aktiv eintippen muss. Eine wichtige Teilaufgabe ist die Recherche und das Versehen von datenschutzrechtlichen Aspekten, die bei der Datenspeicherung und der Implementierung von \textbf{Instant Messaging Systemen} bestehen.

\section{Aufgabenstellung}
\label{sec:Aufgabenstellung}

Die konkrete Aufgabe ist es, einen Chatserver auf Basis des \textbf{\ac{XMPP}} in Betrieb zu nehmen über den sich mehrere Chat-Clients authentifizieren und verschlüsselt Nachrichten austauschen können. Durch eine gründliche Recherche soll eruiert werden, welcher Chatserver sich hierfür eignet und warum dieser Chatserver für das Projekt verwendet wird. Der Nachrichtenaustausch soll Ende-zu-Ende verschlüsselt erfolgen. Eine wichtige Teilaufgabe ist, dass die Software Gruppenchats verschlüsselt unterstützt. Eine Zustellung der Nachrichten in Echtzeit soll implementiert werden. Benutzer sollen Nachrichten über eine Web-Oberfläche eingeben und empfangen können. Es soll ein Prototyp eines Machine Learning Algorithmus der Kategorie \textbf{\ac{NLP}} implementiert werden und in die Web-Oberlfäche integriert werden. Eine wichtige Aufgabe ist, dass das gesamte Chatsystem datenschutzfreundlich implementiert und programmiert wird.

\section{Ziele der Arbeit}
\label{sec:Ziele}

Die Ziele der Arbeit sind es, eine geeignete Netzwerkumgebung und einen lauffähigen Chatserver auf Basis von \ac{XMPP} in Betrieb zu nehmen. Es sollen tiefe Kenntnisse und Erfahrungen mit dem Umgang des \ac{XMPP} Protokolls gesammelt werden und sich mit dem Aufbau des Nachrichtenprotokolls auseinandergesetzt werden. Außerdem sollen sich mit der Arbeitsweise und den Grundlagen von \textbf{Instant Messaging Systemen} und der Nachrichtenübertragung in Echtzeit vertraut gemacht werden. Es soll ein funktionsfähiger Prototyp einer WebUI entstehen, der einen \ac{MUC} unterstützt und über den Chatserver Nachrichten verschlüsselt versendet und empfängt. Durch Recherche und praktische Entwicklungen soll das Fachwissen in der Programmiersprache \textbf{Python} vertieft werden. Zusätzlich sollen Fähigkeiten im Bereich \textbf{Machine Learning} erlernt werden und ein lauffähiger Prototyp eines \ac{NLP}-Models in die Weboberfläche integriert werden. Das letzte Ziel beinhaltet, Kenntnisse im Bereich des Datenschutzes bei Chatapplikationen zu erarbeiten und den Chatserver datenschutzfreundlich zu konfigurieren.

\section{Stand der Technik}
\label{sec:StandDerTechnik}

\textbf{Instant Messaging Systeme} existieren in der heutigen Form seit Ende der 1990er Jahre, ausgehend von einer Öffnung des Internets für einen größeren Nutzerkreis außerhalb von Forschungsinstitutionen. Das erste \ac{IMS}, welches eine größere Verbreitung fand, war \textbf{\ac{ICQ}} der Firma Mirabilis, welches Benutzern ermöglicht in einer grafischen Oberfläche Nachrichten untereinander oder in Chatrooms auszutauschen \cite{ICQ}. Die Systeme werden in unterschiedlichen Ausprägungen stets weiterentwickelt. Es kommen zur grundlegenden Funktion des Nachrichtenaustauschs weitere Features, wie Dateiübertragung, \textbf{\ac{VoIP}}, Video over IP, \textbf{Ende-zu-Ende-Verschlüsselung} oder Sprachnachrichten hinzu \cite{gross2007}. 
Weitere typische Funktionen sind die Möglichkeit des Logins mit Passwort und die Angabe der Verfügbarkeit oder die Übermittlung seines Standorts an andere Benutzer. Die Ansprüche an Funktionalitäten von Chatsystemen und der Echtzeitübertragung wachsen stetig weiter und es ist zu erwarten, dass in den kommenden Jahren der Fortschritt die Echtzeitübertragung und die Funktionalitäten von \textbf{Instant-Messaging-Systeme} verbessern wird. Um einen Beitrag zur Steigerung der Präsenz und der Funktionalitäten zu leisten, soll ein Chatsystem gebaut werden, dass den Anforderungen bisheriger Chatsysteme entspricht. Das Chatsystem soll die Datenschutzbestimmung erfüllen und Benutzerdaten nur speichern, wenn dies für die Bereitstellung des Dienstes nötig ist.\cite{anastasiaIMS}
\newpage

\chapter{Datenschutzrechtliche Aspekte}
\label{chap:Datenschutz}

\newpage

\chapter{Anforderungen}
\label{Anforderungen}

\newpage

\chapter{Theoretische Grundlagen}
\label{chap:Theorie}

\section{Funktionsweise von Instant-Messaging-Systemen}
\label{sec:IMSFunktion}

Verschiedene \ac{IMS} besitzen ähnliche Funktionsweisen. Möchte ein Benutzer einen \ac{IMS} nutzen, muss dieser eine bestimmte Software installieren. Diese Software wird als \textbf{Instant-Messaging-Client (IM-Client} bezeichnet. Der Benutzer muss sich an einem Authentifikationsserver registrieren. Es wird ein Benutzerprofil angelegt, welches aus aus einem Benutzernamen (User ID) und einem Passwort (Pre-Shared-Key) besteht. Diese Daten werden beim Login des Benutzers vom \ac{IMS} abgefragt. Ein Benutzerprofil kann weitere personenbezogenen Daten bestehen, wie zum Beispiel, Wohnort, Geschlecht und Geburtsdatum. Die Anzahl der Server für die Authentifizierung und der Bereitstellung des Dienstes variiert je nach der Anforderung an Verfügbarkeit und Anzahl der zu erwartenden Nutzer des \ac{IMS}. Die horizontale Skalierung von Servern und Diensten unterliegt allein dem Betreiber dieser Systeme, häufig finden sich jedoch Cluster-Systeme mit intelligenter Lastverteilung (Load-Balancing). \autoref{img:StrukturIMS} zeigt den allgemeinen Aufbau eines \ac{IMS}. In der Abbildung wird eine horizontale oder vertikale Skalierung des \ac{IMS} vernachlässigt. Der Authentifikationsserver symbolisiert einen oder mehrere Server.\cite{anastasiaIMS}

\begin{figure}[h]
	\centering
	\includegraphics{images/GrundlegendeStrukturIMS}
	\caption{Allgemeine Struktur eines Instant-Messaging-Systems}
	\label{img:StrukturIMS}
\end{figure}

Hat sich der Benutzer mit seinen Login-Daten erfolgreich gegenüber dem Authentifikationsserver authentifiziert, kann er andere, bereits registrierte, Benutzer kontaktieren. Die Verbindungsinformationen wie zum Beispiel IP-Adresse, die dem Client lokal zugewiesene Portnummer und die Kontaktliste (Freunde des Benutzers) werden an den Präsenzserver übermittelt. Dieser ist abhängig von der Implementierungsform ein Teil des \textbf{IM-Servers} oder ein eigenständiger Server. Der Präsenzserver überprüft, welche Benutzer aus der Kontaktliste verfügbar und angemeldet sind. Der Server sendet dem Benutzer die Ergebnisse der Statusüberprüfung zurück. Ist einer der gewünschten Benutzer \glqq online\grqq (verfügbar) kann durch die Auswahl der entsprechenden Kennung eine Verbindung aufgebaut werden und der Nachrichtenaustausch kann beginnen. Dies ist möglich, weil der Server dem Sender die IP-Adresse und die Ziel-Portnummer des Kommunikationspartners übermittelt. Die Nachrichten werden direkt zwischen den Clients übertragen oder über den Server zwischen den Kommunikationspartnern übermittelt. Die Art der Übertragung ist von System zu System je nach Realisierung unterschiedlich. Der Kommunikationspfad ist abhängig von der Architektur und dem verwendeten Protokoll. Eine Nachricht kann zentral über den Server vermittelt werden (vgl. \autoref{img:StrukturIMS}-indirekte Nachricht) oder nach dem Peer-to-Peer-Prinzip (vgl. \autoref{img:StrukturIMS}-direkte Nachricht) erfolgen. 

\section{XMPP-Extensible Messaging and Presence Protocol}
\label{sec:XMPP}
\ac{XMPP} bedeutet \textbf{Extensible Messaging and Presence Protocol}. Wird dieses ins Deutsche übersetzt, so entsteht die Bedeutung eines erweiterbares Nachrichten- und Anwesenheitsprotokoll. Eine Definition die \ac{XMPP} sehr gut beschreibt. \ac{XMPP} basiert auf \ac{XML}, welches eine Markup Sprache darstellt. Das Ziel von \ac{XMPP} war ein Protokoll für das Instant Messaging zu entwickeln. Laut dem RFC6120 lässt sich mittels \ac{XMPP} Daten zwischen zwei oder mehreren Netzwerkeinheiten nahezu in Echtzeit austauschen, welches als Vorteil für Sofortnachrichten bezeichnet werden kann. Diesbezüglich nutzt es das Internet und erlaubt den Usern Sofortnachrichten an andere Anwender innerhalb des Internets zu schicken. \ac{XMPP} lässt sich in viele verschiedene Funktionen aufteilen, weshalb der grundlegende Zweck ein anderes Ziel verfolgt. Im einfachsten Sinne ist die Idee von \ac{XMPP} den Austausch von kleinen Teilen strukturierter Daten (\glqq \ac{XML} stanzas\grqq) zwischen einem oder mehreren Netzwerkteilnehmern zu ermöglichen. Primär wird \ac{XMPP} mithilfe einer Client-Server-Architektur implementiert, bei der sich ein Client mit einem Server verbindet, um mit anderen Teilnehmern Daten auszutauschen. Anderseits kann es als Protokoll auch zwischen Servern fungieren. Daraus resultiert der Vorteil nahezu unabhängig von Betriebssystemen und Browsern zu sein. Wird die Client-Server-Architektur implementiert, so ist in der Regel der Ablauf definiert durch folgende Schritte: \cite{RFC6120} 
\begin{enumerate}
	\item Bestimmen der IP-Adresse und des Ports zu dem sich verbunden werden soll 
	\item Eine \ac{TCP} Verbindung öffnen/aufbauen
	\item Öffnen eines \ac{XML}-Streams über \ac{TCP}
	\item Optional: Verwendung von \ac{TLS} für die Verschlüsselung
	\item Verwendung des SASLs Frameworks für die Authentifizierung
	\item Eine Ressource an den \ac{XML} stream anbinden
	\item Austausch unbegrenzter \glqq \ac{XML} stanzas\grqq ()=> kleine Teile strukturierter Daten) mit anderen Netzwerkteilnehmern
	\item Schließen des \ac{XML} streams
	\item Schließen der \ac{TCP} Verbindung
\end{enumerate}
Die folgende \autoref{img:XMPPcommunication} zeigt den Start einer Kommunikation, wie im genannten Ablauf dargestellt, über \ac{XMPP} und den \ac{XML}-streams.

\begin{figure}[h]
	\centering
	\includegraphics[width=\textwidth]{images/XML_Wireshark}
	\caption{Kommunikationsaufbau des \ac{XMPP}-Protokolls}
	\label{img:XMPPcommunication}
\end{figure} 
\newpage
Für den Austausch von Daten gibt es zwei elementare Konzepte. Zum einen die \ac{XML}-Streams und die \ac{XML}-Stanzas. Diese zwei Konzepte werden definiert um das Verständnis des Datenaustauschs, wie es im obigen Ablauf definiert ist, zu erlangen. Bei dem Austausch der Daten mittels \ac{XML} streams wird von einem Container zwischen den Teilnehmern gesprochen. Der \ac{XML} stream ist durch den \glqq stream header\grqq (z.B. \ac{XML} <stream>) und dem Ende des streams, dargestellt durch \ac{XML} </streams>, eindeutig definiert. Die Anzahl der austauschbaren \ac{XML} Elemente ist unbegrenzt und durch die Lebensdauer des streams definiert. Das \ac{XML} stanza wird nun als diskrete semantische Einheit strukturierter Daten, das von einem Teilnehmer zu einem anderen über den \ac{XML} stream gesendet wird bezeichnet. Das \ac{XML} stanza ist ein direktes Kind-Element des streams. Ein stanza kann wiederum selbst Kind-Elemente enthalten, das den \ac{XML} stream definiert. Im Kern fungiert der \ac{XML} stream wie eine Hülle um alle \ac{XML} stanzas, die während einer Session versendet werden. Ein Aufbau kann wie in der folgenden \autoref{img:StrukturXMLstream} repräsentiert werden. \cite{RFC6120Sec4}
\begin{figure}[h]
	\centering
	\includegraphics[scale=1.2]{images/XML_Stream}
	\caption{Allgemeine Struktur eines XML streams}
	\label{img:StrukturXMLstream}
\end{figure}
\newpage


\section{Python}
\label{sec:Python}
\newpage

\section{Ejabberd}
\label{sec:ejabberd}
Ejabberd ist einer der bekanntesten freien \ac{XMPP}-Server auf der Welt und kann in vielerlei Hinsicht verwendet werden. Sowohl Großprojekte als auch kleine Instanzen machen sich die Eigenschaften von ejabberd zum Vorteil. Der Start von ejabberd ist dem Jahr 2002 zuzuordnen. Ejabberd ist eine Abkürzung und steht für \glqq Erlang Jabber Daemon\grqq. Wie die Definition zeigt bezieht sich ejabberd auf die Programmiersprache Erlang. Grund hierfür ist, dass die ejabberd Software in Erlang geschrieben ist. Seit dem Start wurde es von Grund auf für die Unternehmensbereitstellung entwickelt, vor allem mit dem Ziel robust zu sein. Aufgrund davon, dass der Fokus auf die Unternehmen lag, war es wichtig die Fehleranfälligkeit von ejabberd zu minimieren. Ein Vorteil der sich bis zum heutigen Zeitpunkt bewahrt hat. Außerdem kann ejabberd die Ressourcen mehrerer geclusterter Systeme nutzen. Des Weiteren besitzt ejabberd die Eigenschaft der Skalierbarkeit, indem die Kapazitäten mit wenig Aufwand erhöht werden kann. Während der Entstehungsphase war das \ac{XMPP}-Protokoll, welches in \autoref{sec:XMPP} beschrieben wird, noch unter dem Namen Jabber bekannt. Ejabberd kann verschieden benutzt werden. Im Fall der Studienarbeit wird die Community Edition von ejabberd benutzt, welche als open source zur Verfügung steht. Neben der Community Edition gibt es auch noch Möglichkeiten einer Business Edition, welches vor allem für die großen Unternehmen mit besserem Support rund um die Uhr und größeren Funktionen konzeptioniert ist. Die Architektur eines ejabberd services erweitert die Kernfunktionen von \ac{XMPP}, welches das Senden von Nachrichten ist, um Faktoren wie die Skalierbarkeit, Konfigurierbarkeit und Fehlertoleranz. Außerdem gilt die Architektur von ejabberd als Modular. Das bedeutet, dass es an den Zweck eines Projektes angepasst werden kann. Die Modularität bringt bspw. Funktionen wie Gruppenchat mit ein. Aufgrund der großen Anzahl an Modulen werden lediglich die Module, die für das Projekt relevant sind aufgelistet.
Relevante Funktionen:
\begin{itemize}
	\item Einzelchat
	\item Gruppenchat (auch als Multi-User-Chat (\ac{MUC}) bezeichnet)
	\item Offline Nachrichten 
	\item Web-Unterstützung
	\item Nachrichtenübermittlungsbestätigung
\end{itemize}
Eine wichtige Eigenschaft ist das Authentifizieren von Nutzern, welches ebenfalls von ejabberd unterstützt wird. Dafür kann ejabberd sowohl mit einer internen, als auch mit einer externen Datenbank zusammenarbeiten. Aufgrund der genannten Eigenschaften und Funktionen von ejabberd, werden eine Vielzahl an mögliche Anwendungsgebieten abgedeckt. Während für viele kleine Projekte die interne Datenbank Mnesia ausreichend ist, wird im Rahmen der Studienarbeit mit einer externen Datenbank gearbeitet.\cite{EjabberdDoc}

\section{Chatserver}
\label{sec:Chatserver}

\begin{lstlisting}[language=Python,caption=Example Listing Python,label={lst:Example}]
"""The first step is to create an SMTP object, each object is used for connection 
with one server."""

import smtplib
server = smtplib.SMTP('smtp.gmail.com', 587)

#Next, log in to the server
server.login("youremailusername", "password")

#Send the mail
msg = "
Hello!" # The /n separates the message from the headers
server.sendmail("you@gmail.com", "target@example.com", msg)
\end{lstlisting}


\section{Datenbank}
\label{sec:Datenbank}
Im folgend Abschnitt liegt der Augenmerk auf die theoretischen Grundlagen einer MySQL-Datenbank. Grund hierfür ist die Verwendungen einer MySQL-Datenbank um Informationen über die User zu speichern. Damit wird auch die Definition, eine Datenbank beinhaltet eine Sammlung von mehrerer Daten, die sich aufeinander beziehen können, und von einem \ac{DBMS} verwaltet werden, beschrieben. Relevant für den Zugriff bildet die einheitliche logische Schnittstelle zum \ac{DBMS}. Der Zugriff kann mit dem weitverbreiteten Standard \ac{SQL}, welches im allgemeinen Sinn eine Programmiersprache darstellt, erfolgen. Dafür wird der Datenbank über \ac{SQL} nur mitgeteilt welche Daten benötigt werden. Ein Datenbanksystem muss wesentliche Eigenschaften besitzen, um den Zweck Daten zu speichern und sie wieder abzufragen, zu erfüllen. Nachfolgend sind solche Eigenschaften aufgeführt:
\begin{itemize}
	\item Unabhängigkeit von dem physischen Aufbau
	\item Schutz bei (Mehrfach-) zugriff auf Daten
	\item Integrität
	\item Zuverlässigkeit
	\item Ausfallsicherheit
\end{itemize}
Eine Datenbank kann entsprechend dem zugehörigen Modell definiert werden. Die Datenbankmodelle unterteilen sich in relationale, objektorientierte, hierarchische und netzwerkartige sowie in moderne Datenbanken. Aufgrund der großen Anzahl, wird lediglich auf die das relationale Modell eingegangen, da diesem MySQL zugeordnet werden kann, welches in dieser Studienarbeit angewendet wird. Dementsprechend sind wichtige Schlüsselbegriffe zu definieren. Eine Tabelle wird ebenfalls als Relation bezeichnet. Eine Zeile der Relation heißt Tupel, die Spalten Attribute. Die Kardinalität entspricht der Anzahl von Tupeln und der Grad bezieht sich auf die Anzahl der Attribute. Ein wesentliche Bestandteil einer relationalen Datenbank ist der Primärschlüssel, welche zwingend erforderlich ist und eindeutig sein muss. Der Schlüsselbegriff Gebiet bezieht sich auf ein Definitionsbereich eines Attributes, welches alle gültigen Werte diesen Attributes umfasst. Ein Beispiel für eine Relation mit allen relevanten Schlüsselbegriffen ist in der folgenden \autoref{img:RelationDB_Begriffe} dargestellt.\cite{Schicker2017}

\begin{figure}[h]
	\centering
	\includegraphics[scale=1.2]{images/RelationMitSchluesselbegriffe}
	\caption{Relation mit Schlüsselbegriffe}
	\label{img:RelationDB_Begriffe}
\end{figure}
Demzufolge ist eine relationale Datenbank definiert als eine Datenbank mit einer Ansammlung von zeitlich variierenden, normalisierten Relationen mit passenden Graden.
MySQL ist eins der weitverbreitetsten Datenbankenverwaltungssystemen, mitunter weil die Software open source und somit von jedem benutzt werden kann. Entwickelt wird MySQL von der Oracle Cooperation. Die MySQL-Datenbanken sind relational und orientieren sich an den oben genannten Schlüsselbegriffen, die eine relationale Datenbank definieren. Die MySQL-Datenbank ist in der Software eine Client-Server-System, das ,wie es offiziell von MySQL beschrieben wird, aus einem Mehr-Thread-SQL-Server besteht. Dadurch werden verschiedene Backends, Clientprogramme und Verwaltungswerkzeuge unterstützt.\cite{MysqlDoc}
\newpage

\section{Grafische Benutzeroberfläche}
\label{sec:Benutzeroberfläche}
Eine Grafische Benutzeroberfläche ist im Detail eine Schnittstelle zum Nutzer eines Computers. Die im Hintergrund agierende Software soll mittels Symbole und anderen Elementen bedienbar gemacht werden. Im Englischen wird die Benutzeroberfläche als \ac{GUI} bezeichnet, welches im weitere Verlauf ebenfalls als Bezeichnung verwendet wird. Allgemein besitzt eine \ac{GUI} Bedien- und Steuerelement die mit der im Hintergrund laufenden Software interagiert. Der Einsatz einer \ac{GUI} liegt an der Usability, welches die Benutzerfreundlichkeit beschreibt. Eine \ac{GUI} soll demnach dem User die Möglichkeit geben, die Software auch ohne großen Aufwand bedienen zu können. Eine Eigenschaft die heutzutage nahezu jedes Programm beinhalten muss. %TODO
\url{https://de.wikipedia.org/wiki/Grafische_Benutzeroberfl%C3%A4che}
	
Im Rahmen der GUI-Programmierung müssen mehrere Aspekte berücksichtigt werden um die Usability zu erfüllen. Dafür werden die Aspekte in 8 goldene Regeln zusammengefasst, die bei der Entwicklung einer GUI berücksichtigt werden sollten.
\begin{enumerate}
	\item Strebe nach Konsistenz
	\item Sorge für universelle Bedienbarkeit
	\item Biete informative Rückmeldung
	\item Entwerfe abgeschlossene Dialoge
	\item Biete einfache Fehlerbehandlung
	\item Lass die Einfache Umkehrung von Aktionen zu
	\item Vermittle ein Gefühl der Kontrolle
	\item Entlaste das Kurzzeitgedächtnis
\end{enumerate}
Im Folgenden wird kurz auf die 8 Regeln eingegangen, damit eine Verbindung zur Umsetzung der \ac{GUI} in \autoref{subsec:EntwicklungGUI} entsteht.

Der erste Punkt, \textbf{Strebe nach Konsistenz}, befasst sich bspw. mit dem Erscheinungsbild einer Seite. Die Seite sollte dem Aspekt zur Folge die Elemente, wie einen Button, immer an der gleichen Stelle positionieren.

Der nächste Punkt, \textbf{Sorge für universelle Bedienbarkeit}, beinhaltet Anforderungen an die Vielfalt an Nutzer. Somit sollte jeder Nutzer die Bedienung, Kürzel usw. ohne Einschränkungen verstehen können.

Ein Nutzer verlangt nach ausreichendem Feedback, die vor allem informativ und in vielen Fällen auch hilfreich sind. Ein Fakt der im dritten Punkt, \textbf{Biete informative Rückmeldung}, behandelt wird

Der vierte Punkt, \textbf{Entwerfe abgeschlossene Dialoge}, umfasst die Führung eines Nutzers, bspw. durch eine Verkaufsprozess. Zu jedem Zeitpunkt soll der Nutzer wissen, in welchem Prozess er sich befindet.

Um die Bedürfnisse der Nutzer gerecht zu werden muss wie in Punkt fünf, \textbf{Biete einfache Fehlerbehandlung}, definiert, die Fehler mit nützlichen Informationen für den Nutzer ausgestattet werden. Außerdem muss ein Ausweg in den normalen Programmablauf gewährleistet sein.

Ein Nutzer soll mit Punkt sechs, \textbf{Lass die Einfache Umkehrung von Aktionen zu}, die Möglichkeit haben kleine Ereignisse rückgängig zu machen.

Der siebte Punkt, \textbf{Vermittle ein Gefühl der Kontrolle}, beinhaltet die Aktion und Reaktion. Demnach soll die Software auf eine Aktion des Nutzers reagieren und nicht andersrum. 

Der letzte Punkt, \textbf{Entlaste das Kurzzeitgedächtnis}, soll gewährleisten, dass die GUI schlicht und einfach gehalten werden soll. Ein Nutzer sollte sich keine große Anzahl an Informationen merken müssen um einen Aktion auszuführen.
\cite{UI8Regeln2019}

Aufgrund der 8 goldenen Regeln, kann im Umsetzungsprozess auf diese zurückgegriffen werden um eine benutzerfreundliche Oberfläche zu erstellen. Im folgenden Abschnitt wird auf die \ac{GUI}-Programmierung mit Python eingegangen.


\subsection{GUI-Programmierung mit Python}
\label{subsec:GuiPython}
Mit Python als Programmiersprache kommen mehrere Module, welche für die Grafische Benutzeroberfläche benutzt werden können, einher. Dementsprechend wird in diesem Abschnitt ein Einblick in die Möglichkeiten, passend zur Aufgaben- und Problemstellung, gegeben werden. Abhängig von den Erkenntnissen folgt eine Entscheidungsmatrix um das passende Modul für das Projekt auszuwählen.

Die \ac{GUI}-Programmierung bei Python unterscheidet sich im Konzept nicht allzu sehr von anderen Programmiersprachen. Neben Schaltflächen stehen auch Fenster oder Menüs als Komponenten zur Verfügung. Angeordnet können diese in sogenannten Container. Für eine geeignete Anwenderschnittstelle müssen die Komponenten in die Container integriert werden. Hinzukommt ein äußerer Container, auch als Frame bezeichnet, ein Layout-Manager und eine Implementierung damit Aktionen durch Komponenten ausgelöst werden können. Wie im \autoref{sec:Benutzeroberfläche} erläutert, verfolgt eine \ac{GUI} mehreren Regeln. Demnach gibt es für die Programmierung einer grafischen Benutzeroberfläche vielerlei Möglichkeiten. In Python werden \ac{GUI} Frameworks angeboten, die eine Programmierung erleichtern sollen. Diese Frameworks können sich in der Art und der Verwendung unterscheiden. Aufgrund davon muss erst die Art des Frameworks ermittelt werden. Dafür wird eine Entscheidungsmatrix erstellt, basierend auf Faktoren, die passend zum Projekt Auswirkungen haben.\cite{Steyer2018} \\
Die nachfolgende \autoref{tab:EntscheidungsmatrixFrameworkart} spiegelt die Ergebnisse eines Vergleichs zwischen den Tool-Frameworks und den Web-Frameworks von Python wieder. Die Kriterien orientieren sich am Projekt und werden nachfolgend erläutert. Die \textbf{Nutzerfreundlichkeit} umfasst wie verständlich ein GUI mithilfe des Tools aufgebaut werden kann. Außerdem beinhaltet das Kriterium auch die Möglichkeiten des Tools auf die Nutzerfreundlichkeit einzugehen. Die \textbf{Schwierigkeit} orientiert sich an der Implementierung. Dabei geht darum wie leicht, schnell und komfortabel die Frameart integriert werden kann. Unabhängig von der Schwierigkeit geht es bei der \textbf{Umsetzbarkeit} um die Mittel, welche für eine Umsetzung benötigt werden, sowie auch um den Aufwand bei der Umsetzung. Die \textbf{Kompatibilität} umfasst die Unabhängigkeit von Betriebssystemen, Browsern usw. Standard für eine grafische Oberfläche bildet auch das \textbf{Design}. Demnach geht es um die Möglichkeiten eines guten Erscheinungsbilds der Oberfläche mit dem entsprechenden Framework. Das letzte Kriterium, \textbf{Verwendungszweck}, bildet die Verbindung zum Projekt. Dabei wird analysiert wie die Frameworkart zum Zweck des Projektes passt. Die Gewichtung der Kriterien innerhalb der \autoref{tab:EntscheidungsmatrixFrameworkart} sind der TABELLE des ANHANGS zu entnehmen.
\renewcommand{\arraystretch}{2}
\begin{table}[h]
\centering
	\begin{tabular}{l|c|c|c}
		& & \multicolumn{2}{c}{Arten einer Python-GUI} \\
		Kriterien & Gewichtung & Tool-Framework & Web-Framework \\
		\hline
		Nutzerfreundlichkeit & 33\% & 2 & 2 \\
		\hline
		Schwierigkeit & 7\% & 2 & 1,7  \\
		\hline
		Umsetzbarkeit & 13\% & 2 & 2,2\\
		\hline
		Kompatibilität & 13\% & 2,2 & 1,7 \\
		\hline
		Design & 7\% & 2 &  2\\
		\hline 
		Verwendungszweck & 27\% & 2,5 & 1,5 \\
		\hline
		\textbf{Gesamtwertungszahl} & \textbf{100\%} & \textbf{2,161} & \textbf{1,831} \\
	\end{tabular}
\caption{Entscheidungsmatrix der Frameworkart}
\label{tab:EntscheidungsmatrixFrameworkart}
\end{table} \\
Die \autoref{tab:EntscheidungsmatrixFrameworkart} bildet eine Hilfestellung zur Entscheidung einer passenden Frameworkart für das Projekt. Die Bewertung der Tools entsprechend der Kriterien folgt dem Benotungsschema der akademischen Bildung. Demnach ist eine 1 sehr gut und eine 6 ungenügend. Im Kriterium Nutzerfreundlichkeit schließen beide Frameworks mit der gleichen Note ab, da die Nutzerfreundlichkeit abhängig vom Designstil ist und bei beiden Frameworks umgesetzt werden kann. Die Schwierigkeit, wie das Kriterium in der Einführung definiert wurde, bekommt beim Web-Framework eine bessere Bewertung. Grund hierfür ist, dass für das Web-Framework überwiegend keine aufwändigen Module in Python benötigt werden. Die Gestaltung erfolgt nämlich über HTML und CSS, welche dann mit einem schlichten Modul verknüpft bzw. aufgerufen werden können. Aufgrund von Vorkenntnissen sowohl in HTML als auch in CSS wird die Schwierigkeit für das Web-Framework besser bewertet. Bei der Umsetzbarkeit steht das Tool-Framework aufgrund der Verwendung eines gesamten Moduls besser da. Ein Web-Framework bringt HTML, CSS und das Modul mit was ein größeren Umfang bedeutet. Vor allem muss der Programmierer mit allen Elementen umgehen können, weshalb eine Umsetzung schwieriger sein könnte. In der Kompatibilität wird das Web-Framework aufgrund der Unabhängigkeit bezüglich des Betriebssystem besser bewertet. Außerdem kann eine ausgelieferte HTML-Seite von nahezu allen Browsern gelesen werden, und eine Möglichkeit die Lösung auf mobilen Endgeräten zu benutzen ist ebenfalls enthalten. Im Design unterscheiden sich die Frameworkarten nicht. Beide bieten vielerlei Möglichkeiten im Bezug auf die Gestaltung. Aufgrund des Projektes einen Chat über \ac{XMPP} und ejabberd zu realisieren wird schon hauptsächlich das Internet genutzt. Dadurch bietet es sich den Server eine HTML-Seite ausliefern zulassen. Aus diesem Grund wird das Web-Framework im Rahmen dieses Kriteriums besser bewertet. Anhand der Gesamtwertungszahl entsprechend der Benotung und der Gewichtung sind beide Frameworks zu empfehlen, dennoch schließt das Web-Framework besser ab, weshalb im weiteren Verlauf ein Web-Framework zum Einsatz kommen soll. \cite{FrameworkOverview} \cite{WebFramework}\\
Mit diesem Ergebnis müssen die verschiedenen Möglichkeiten eine Web-Framework in Bezug auf Python analysiert und bewertet werden. Auch in diesem Fall liefert eine Entscheidungsmatrix ein aufschlussreiches Ergebnis. Die Definition der selben Bewertungskriterien sind der Einführung der \autoref{tab:EntscheidungsmatrixFrameworkart} zu entnehmen. Kriterien wie Umfang und Komplexität werden in die Betrachtung miteinbezogen. Der \textbf{Umfang} spiegelt die Möglichkeiten und Größe im Aufbau des Tools wieder. Die \textbf{Komplexität} befasst sich im Grunde mit dem Aufwand einer Implementierung sowie auch die Hilfestellungen durch geeignete Dokumentationen. Das Bewertungsschema orientiert sich ebenfalls an der \autoref{tab:EntscheidungsmatrixFrameworkart}.\\
\begin{table}[h]
	\centering
	\begin{tabular}{l|c|c|c|c|c}
		Kriterien & Gewichtung & Django & Flask & CherryPy & Bottle\\
		\hline
		Umfang & 17\% & 1,5 & 2 & 2 & 2,5 \\
		\hline
		Schwierigkeit & 17\% & 2,5 & 1,7 & 2,2 & 2 \\
		\hline
		Komplexität & 33\% & 2,2 & 1,7 & 2 & 1,7 \\
		\hline
		Verwendungszweck & 33\% & 2 & 2 & 2 & 2,2 \\
		\hline
		\textbf{Gesamtwertungszahl} & \textbf{100\%} & \textbf{2,067} & \textbf{1,850} & \textbf{2,033}  & \textbf{2,050} \\
	\end{tabular}
	\caption{Entscheidungsmatrix eines Web-Frameworks}
	\label{tab:EntscheidungsmatrixWebFramework}
\end{table}Die Gesamtwertungszahl, welches der \autoref{tab:EntscheidungsmatrixWebFramework} entnommen werden kann, liefert ein knappes Ergebnis. Demnach bietet sich Flask am besten als Framwork einer grafischen Oberfläche an. Im Umfang setzt sich das Tool Django gegen die Konkurrenten durch, da es aufgrund seiner Größe vielerlei Möglichkeiten mitbringt. Während die anderen Tools überwiegend als Mikroframework bezeichnet werden zeichnet sich Django als Full-Stack/high-level Framework aus. Ebenfalls ein Grund weshalb die Schwierigkeit bei Django mit einer 2,5 bewertet wurde. Ein Mikroframework bildet in der Hinsicht die leichtere und schnellere Variante. Die Komplexität wurde vor allem bei Flask und Bottle gut bewertet. Grund hierfür ist unter anderem eine geringere Komplexität durch Möglichkeiten einer schlichten Implementierung. Außerdem bietet bspw. Flask eine sehr gute Dokumentation, welche die Komplexität verringern. Der Verwendungszweck ist nahezu gleichbleibend, da alle Frameworks genutzt werden können um das Projekt umzusetzen. Ausschließlich bei Bottle ist überwiegend die Rede von einem Framework das vor allem zur Einführung in das Themengebiet der Programmierung grafischer Oberflächen mit eine Web-Framework benutzt wird. Anhand des Ergebnisses und der sehr guten Dokumentation wird Flask als Web-Framework zum Einsatz kommen.\cite{BottleDoc} \cite{CherryPyDoc} \cite{DjangoDoc} \cite{FlaskDoc}

\subsection{Flask}
\label{subsec:Flask}
Wie unter \autoref{subsec:GuiPython} beschrieben, wird im Rahmen der Studienarbeit Flask verwendet. Es ist ein Python Web-Framework welches eine Schnittstelle zum Webserver bildet. Bekannt ist Flask durch die Eigenschaft schnell und leicht eine minimale Anwendung zu implementieren und diese bei Bedarf in der Größe zu skalieren. Für die Verwendung von Flask sind zwei Bibliotheken wichtig. Bei diesen handelt es sich zum einen über die \ac{WSGI}-Bibliothek Werkzeug und der Template-Bibliothek Jinja2. Module die standardmäßig von Flask verwendet werden. Flask bildet im Grunde das Bindeglied zwischen den Modulen und erlaubt es bspw. mithilfe des \textbf{Routings} an bestimmten Endpunkten eine \ac{HTML}-Datei zu rendern. Ein Möglichkeit dafür zeigt das folgende \autoref{lst:FlaskMinApp}.
\begin{lstlisting}[language=Python,caption=Example Listing of Flask Python,label={lst:FlaskMinApp}]
from flask import Flask
app = Flask(__name__)

@app.route('/')
def homepage():
	 return render_template('homepage.html')
\end{lstlisting}
Das \autoref{lst:FlaskMinApp} spiegelt das Routen sowie das Rendern einer \ac{HTML}-Datei von Python wieder. Aufgrund des Paketverwaltungsprogramm pip kann Flask komfortable installiert werden. Dementsprechend wird Flask in der ersten Zeile importiert, dann einer Anwendung zugeordnet und eine Route definiert. Nun besitzt Flask vielerlei Möglichkeiten, so kann bspw. je nach Anfrage (POST, GET, usw.) differenziert werden. In dem Zusammenhang kommt eine weiterer möglicher Vorteil von Flask zum Einsatz. Dieser beinhaltet die Möglichkeit einen Webserver, den Flask zur Verfügung stellt, im Entwicklungsmodus zu starten und aktuelle Code-Zeilen zu testen.
Flask ist also nicht zuständig für die Gestaltung der grafischen Oberfläche sondern benutzt sogenannte \textbf{Templates} in Form von HTML-Datei, die eine Webseite aufbauen. Für eine dynamische Webanwendung und den gestalterischen Aspekten sind statische Dateien, wie z.B. \ac{CSS}, notwendig. Aus diesem Grund benötigt Flask eine vorgegebene Ordnerstruktur, welche der nachfolgenden \autoref{img:TreeFlaskOrdnerstrukt} entnommen werden kann.
\begin{figure}[h]
	\centering
	\includegraphics[width=\textwidth]{images/TreeFlaskOrdnerstrukt}
	\caption{Ordnerstruktur unter Flask}
	\label{img:TreeFlaskOrdnerstrukt}
\end{figure}
Für die grafische Benutzeroberfläche würden diese Elemente ausreichen um eine simple Webseite aufzubauen, mit den \ac{HTML}- und \ac{CSS}-Dateien zu verknüpfen, über das Routen die Seite zu rendern und letztendlich, über den von Flask gestellten Webserver, zu testen. Dennoch bietet Flask mehr Möglichkeiten auch im Zusammenhang des Projektes. Zum Beispiel kann in Flask mit redirects, sessions, cookies, Datenbanken und vieles mehr gearbeitet werden. Dabei verfolgt Flask das Ziel dem Anwender nichts vorzuschreiben, dadurch ist er unabhängiger und kann auf vielerlei Erweiterungen zugreifen. Bei der Anbindung einer \textbf{Datenbank} gibt es so die Möglichkeit bspw. eine MySQL, PostgreSQL, SQLite oder andere Datenbanken zu verwenden. In diesem Zusammenhang bietet Flask das Modul Flask-SQLAlchemy, welches der Flask-Anwendung das SQLAlchemy Paket hinzufügt. Ziel ist es die Verwendung von SQLAlchemy mit Flask zu vereinfachen. Das Paket gilt als \ac{ORM}, und bietet die Möglichkeit high-level Operationen in Datenbank-Kommandos zu übersetzen. Im Rahmen der Studienarbeit wird ebenfalls die Erweiterung Flask-SQLAlchemy angewendet. Demnach kann eine Implementierung, einer minimalen Anwendung mit Datenbank, wie im folgenden \autoref{lst:SQLAlchemyMinApp} dargestellt, aussehen.
\begin{lstlisting}[language=Python,caption=Example Listing of Flask-SQLAlchemy,label={lst:SQLAlchemyMinApp}]
from flask import Flask
from flask_sqlalchemy import SQLAlchemy

app = Flask(__name__)
app.config['SECRET_KEY'] = os.urandom(24) #secret key of app
app.config['SQLALCHEMY_DATABASE_URI'] = config.get('SQLALCHEMY_DATABASE_URI')

db = SQLAlchemy(app)

class User(db.Model):
	user_id = db.Column(db.Integer, primary_key=True, autoincrement=True)
	username = db.Column(db.String(25), unique=True, nullable=False)
	email = db.Column(db.String(35), unique=True, nullable=False)
	passwd = db.Column(db.String(112), nullable=False)
	
	def __init__(self, user, email, passwd):
		self.username = user
		self.email = email
		self.passwd = self.set_password(passwd)
	
	def get_id(self):
		return (self.user_id)

@app.route('/login')
def login():
	user = User(req_content["username"], req_content["eMail"], req_content["password"])
	db.session.add(user)
	db.session.commit()
	return render_template('homepage.html')
\end{lstlisting}
Neben dieser Erweiterung bietet Flask auch ein Modul im Rahmen eines Login, welcher ebenso für das Projekt relevant ist, an. Dafür kann der Login Manager aus Flask-Login verwendet werden. Dieser beinhaltet nützliche Funktionen auf die innerhalb der Anwendung zugegriffen werden kann. Für den Login-Manager werden sessions benötigt, welche ebenfalls importiert werden müssen. Das liegt an der Authentifizierung, weshalb ein sicherer Schlüssel, wie er im \autoref{lst:LoginManagerMinApp} dargestellt ist, erzeugt werden muss.
\begin{lstlisting}[language=Python,caption=Example Listing of Flask-Login,label={lst:LoginManagerMinApp}]
from flask import Flask
from flask_login import LoginManager, login_required, login_user, logout_user, current_user

app = Flask(__name__)

login_mgmt = LoginManager(app)
login_mgmt.login_view = 'login' # name of callback method if unauthorized user accessed a login protected site

@app.route("/login", methods=['GET', 'POST'])
def login():
	if current_user.is_authenticated:
		return redirect(url_for('gochat'))
	login_user(user)
	return render_template('login.html')
\end{lstlisting}
Diese beispielhafte Listings wichtiger Module zeigen die Möglichkeiten des Web-Frameworks. Mitunter der Grund für die Verwendung von Flask. Außerdem sind nur wenige Module dargestellt. Weitere Erweiterungen, die im Rahmen des Projektes eingesetzt werden folgen bei der Umsetzung. Diese dargestellt bilden lediglich die theoretische Grundlagen um das Funktionsprinzip von Flask zu erläutern.

\chapter{Umsetzung und Implementierung}
\label{chap:Umsetzung}

\section{Vorbereitung}
\label{sec:Vorbereitung}

\section{Implementierung des Chatservers ejabberd}
\label{sec:ChatserverEntwicklung}

\subsection{Anbindung des Chatservers}
\label{subsec:Anbindung}

\subsection{Security}
\label{subsec:Security}

\subsection{Konfiguration von ejabberd}
\label{subsec:Konfiguration}

\newpage

\section{Entwicklung des Chat-Clients}
\label{sec:ClientEntwicklung}

\ac{HTML}

\subsection{Entwicklung der GUI}
\label{subsec:EntwicklungGUI}

\subsection{Implemintierung des Backends}
\label{subsec:Backend}

\newpage

\chapter{Datenanalyse}
\label{chap:Datenanalyse}

\newpage

\chapter{Ergebnis}
\label{chap:Ergebnis}

\newpage

\chapter{Fazit}
\label{chap:Fazit}


\end{onehalfspacing}
\newpage

\bibliography{Literatur}
\newpage
%setze Anhang mit appendix
%addpart, sodass Anhang im Inhaltsverzeichnis ohne Buchstaben auftaucht
\appendix
\addpart{Anhang}

\end{document}