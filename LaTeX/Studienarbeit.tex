\documentclass[a4paper,titlepage,halfparskip,12pt]{scrreprt}

\usepackage[ngerman]{babel, varioref}
\usepackage[utf8]{inputenc}
\usepackage[T1]{fontenc}
\usepackage{graphicx}
\usepackage{fancyhdr}
\usepackage{amsmath}
\usepackage{geometry}
\geometry{a4paper, top=25mm,left=25mm,right=25mm,bottom=25mm, footskip=12mm}
\usepackage{longtable}
\usepackage{setspace}
\usepackage{lmodern}
%blocksatz
\sloppy
%formatierung literaturverzeichnisangabe
\bibliographystyle{unsrt}

%Auflistungen von Punkten
\usepackage{paralist} 
%urls anzeigen
\usepackage{url}

%Codelisting
\usepackage{xcolor}
\definecolor{mygreen}{rgb}{0,0.6,0}
\definecolor{mygray}{rgb}{0.5,0.5,0.5}
\definecolor{mymauve}{rgb}{0.58,0,0.82}
\definecolor{burntorange}{rgb}{0.8, 0.33, 0.0}
\definecolor{cornellred}{rgb}{0.7, 0.11, 0.11}

\usepackage{listingsutf8}
\lstset{
commentstyle=\color{mygreen},
numberstyle=\small\color{black},
stringstyle=\color{mymauve},
emph={square}, 
showstringspaces=false,
flexiblecolumns=false,
tabsize=2,
numbers=left,
numberblanklines=false,
stepnumber=1,
captionpos=b,
numbersep=5pt,
xleftmargin=15pt,
breaklines=true,
inputencoding=utf8,
extendedchars=true,
extendedchars=true,
basicstyle=\ttfamily\footnotesize,
keywordstyle = \bfseries\color{burntorange},
keywordstyle = [2]\bfseries\color{cornellred},
literate=%
    {Ä}{{\"A}}1%
    {Ö}{{\"O}}1%
    {Ü}{{\"U}}1%
    {ä}{{\"a}}1%
    {ö}{{\"o}}1%
    {ü}{{\"u}}1%
    {ß}{{\ss}}1,%
frame=single,
frameround=ffff
}

%meta data
\usepackage[hidelinks]{hyperref}
\urlstyle{same}

%akronymverzeichnis
\usepackage[printonlyused]{acronym}

% titel definieren
\newcommand{\titel}{Entwicklung eines Chatsystems\\auf Basis von XMPP}

%autor definieren
\newcommand{\autor}{Lukas Priester,Oliver Klapper}
\newcommand{\keywords}{\autor,\titel,Studienarbeit}

% Allgemeines für das PDF
\hypersetup{
    pdftitle={\titel},
    pdfauthor={\autor},
    pdfcreator={\autor},
    pdfsubject={\titel},
    pdflang={Deutsch},
    pdfdisplaydoctitle=true,
    pdfkeywords={\keywords},
}

% set distances of chapter headlines in document
\renewcommand*\chapterheadstartvskip{\vspace*{20pt}} % set distance to header
% set distance to text
%\renewcommand*\chapterheadendvskip{%
%  \vspace*{1\baselineskip plus .1\baselineskip minus .167\baselineskip}}


\begin{document}

\begin{table}[h]
\centering
\begin{tabular}{lcr}
\includegraphics[height=3.5cm]{images/dhbw-logo}
\end{tabular}
\end{table}
\bigskip
\bigskip
\begin{center}
\vspace*{12mm} {\LARGE\textbf{\titel}}\\
\vspace*{12mm} {\large\textbf{Studienarbeit}}\\
\vspace*{3mm} {\large\textbf{5. - 6. Semester}}\\
\vspace*{12mm} des Studiengangs Informationstechnik (B.Sc.)\\ an der Dualen Hochschule Baden-Württemberg Stuttgart\\
% \vspace*{3mm} an der Dualen Hochschule Baden-Württemberg\\
\vspace*{12mm} von\\
\vspace*{3mm} {\large\textbf{Lukas Priester, Oliver Klapper}}\\
\vspace*{12mm} \today\\
\end{center}
\vfill
\begin{spacing}{1.5}
\begin{tabbing}
mmmmmmmmmmmmmmmmmmmmmmmmmm \= \kill
\textbf{Bearbeitungszeitraum} \> 01.10.2019 - 01.05.2020\\
\textbf{Matrikelnummer, Kurs} \> 7288057, 4191693 \\
\textbf{Kurs} \> TINF17IN\\
\textbf{Betreuer der Hochschule} \> Alfred Becker\\
\textbf{Gutachter der Hochschule} \> Alfred Becker\\
\end{tabbing}
\end{spacing}
%Seitennummerierung ausschalten
\pagenumbering{gobble}
\newpage

\section*{Selbstständigkeitserklärung}

\bigskip

Ich versichere hiermit, dass ich meine Bachelorarbeit (bzw. Studien- und Projektarbeit) mit dem Thema:

\smallskip

%% eigentlich hier \titel verwenden statt duplicated titel, aber umbruch erzwingt
%% doppeltes ausschreiben des titels
\texttt{Entwicklung eines Chatsystems auf Basis von XMPP}

\smallskip

selbstständig verfasst und keine anderen als die angegebenen Quellen und Hilfsmittel benutzt habe.

\bigskip

Ich versichere zudem, dass die eingereichte elektronische Fassung mit der gedruckten Fassung übereinstimmt.*

\bigskip

\begin{small}

* falls beide Fassungen gefordert sind

\bigskip

\bigskip

\noindent\begin{tabular}{ll}
\makebox[2.5in]{\hrulefill} & \makebox[2.5in]{\hrulefill}\\
Ort, Datum & Unterschrift
\end{tabular}
\end{small}

\newpage

%abstract text
\section*{Abstract}

\newpage

%inhaltsverzeichnis
	% Inhaltsverzeichnis
	\cleardoublepage
	\begin{spacing}{1.1}
		\begingroup
		
			% auskommentieren für Seitenzahlen unter Inhaltsverzeichnis
			\renewcommand*{\chapterpagestyle}{empty}
			\pagestyle{empty}
			
			
			%\setcounter{tocdepth}{1}
			%für die Anzeige von Unterkapiteln im Inhaltsverzeichnis
			\setcounter{tocdepth}{2}
			
			\tableofcontents
			\clearpage
		\endgroup
	\end{spacing}

%% new header/footer settings
\renewcommand{\sectionmark}[1]{\markright{\thesection\ #1}} % make header rightmark
\fancypagestyle{fancyheadlines}{
\pagenumbering{arabic}
\fancyhf{}
\lhead{\slshape\rightmark}
%%\rhead{\slshape\nouppercase{\leftmark}}
\renewcommand{\headrulewidth}{0.4pt}
%\lfoot{\slshape DHBW Stuttgart | Lukas Priester, Oliver Klapper}
\cfoot{\thepage}
\renewcommand{\footrulewidth}{0.4pt}
}

% Redefine the plain page style, show only page number in figure,table,...contents
% and chapter pages
\fancypagestyle{plain}{%
  \fancyhf{}%
  %\lfoot{\slshape DHBW Stuttgart | Lukas Priester, Oliver Klapper}%
  \cfoot{\thepage}
  \renewcommand{\headrulewidth}{0pt}% Line at the header invisible
  \renewcommand{\footrulewidth}{0.4pt}% Line at the footer visible
}


\newpage
\pagenumbering{Roman}


%abkürzungsverzeichnis
\cleardoublepage
\addcontentsline{toc}{chapter}{Abkürzungsverzeichnis}
\chapter*{Abkürzungsverzeichnis}
\begin{acronym}[YTMMM]
\setlength{\itemsep}{-\parsep}

\acro{IMS} {Instant Messaging System}
\acro{XMPP} {Extensible Messaging and Presence Protocol}
\acro{NLP} {Natural Language Processing}
\acro{MUC} {Multi User Chat}
\acro{ICQ} {I seek you}
\acro{VoIP} {Voice over IP}
\acro{HTML}{HyperText Markup Language}
\end{acronym}

%abbildungsverzeichnis
\cleardoublepage
\addcontentsline{toc}{chapter}{\listfigurename}
\listoffigures
\newpage
%tabellenverzeichnis
\cleardoublepage
\addcontentsline{toc}{chapter}{\listtablename}
\listoftables
\newpage
%listingverzeichnis
\cleardoublepage
\addcontentsline{toc}{chapter}{\lstlistingname}
\lstlistoflistings
\newpage

\begin{onehalfspacing}

%% header and footer settings
\pagestyle{fancyheadlines}

\chapter{Einleitung}
\label{chap:Einleitung}

Social Media (deutsche Übersetzung: \glqq soziale Medien\grqq{}) ist ein aktuelles und wichtiges gesellschaftliches Thema. Die vielfältigen und breitgefächerten Nutzungsmöglichkeiten beeinflussen das Privat- und das Berufsleben \cite{gabriel2017social}. Gabriel und Röhrs definieren Social Media in \cite{gabriel2017social} als die Verwendung digitaler Medien unter Einsatz computergeschützter Technologien, das heißt, von Hardware- und Softwaresystemen. Sie definieren den Nutzen von Social Media darin, dass Menschen Informationen suchen, erstellen, verteilen und austauschen können. Es gibt eine große Anzahl an unterschiedlichen Definitionen von Social Media. Nach Liu Yinyuan ist Social Media längst wichtiger Bestandteil des Unternehmensmarketings in Deutschland. In seinem Werk \glqq Social Media in China\grqq{} \cite{liu2016social} beschreibt er, dass in Unternehmen nicht mehr über die grundsätzliche Frage debattiert wird, ob Social Media für das Unternehmensmarketing eingesetzt werden soll, sondern wo und wie der Einsatz zielgerichtet erfolgen kann. Heutige Unternehmen sind gekennzeichnet von computergestützten Anwendungssystemen, die in allen Funktionsbereichen zur Planung, Steuerung und Kontrolle der Geschäftsprozesse und zu ihrer Verwaltung eingesetzt werden. Sie setzen dieses System in der B2B-Kommunikation (Business-To-Business-Kommunikation) ein. Immer wichtiger werden \textbf{\ac{IMS}}, wie zum Beispiel Whatsapp, Telegram, iMessage und Jabber, die dazu dienen intern Informationen schnell im Unternehmen zu verbreiten, hoch verfügbar zu machen und um Geschäftsprozesse standortunabhängig steuern zu können \cite{gabriel2017social}. Zusätzlich werden \textbf{Instant Messaging Systeme} auch zur externen Kommunikation benutzt. Nach Gabriel und Röhrs ist es möglich, dass mehrere Unternehmen mit Hilfe von innovativen Kommunikationstechniken zur Erreichung eines gemeinsamen Ziels besser kooperieren können. Außerdem ist die schnelle und direkte Kontaktaufnahme von Kunden über ein Messaging System zum Support eines Unternehmens eine einfache und schnelle Möglichkeit, Fragen zum Produkt ohne langes Warten in der Hotline zu stellen. Nach \cite{b2bmehner} werden Nachrichten von \textbf{Instant Message Systemen} im Gegensatz zu einer E-Mail in Echtzeit übertragen und dem Empfänger direkt zugestellt. Laut einer Statistik von \textbf{statista} benutzen 1,5 Milliarden Nutzer in Deutschland pro WhatsApp pro Monat \cite{statistaIMS}. Der Anteil der Nutzer von Whatsapp in Deutschland beträgt 75 Prozent an der Gesamtbevölkerung \cite{statistaIMS}. Die Statistik zeigt, dass \textbf{Instant Messaging Systeme} eine Möglichkeit in der Zukunft darstellen Kunden direkter anzusprechen oder Support zu gewährleisten. Zusätzlich zu Nachrichtendiensten werden künstliche Intelligenzen und Algorithmen benötigt, die Nutzer unterstützen oder gesammelte Daten von Benutzern charakterisieren oder interpretieren können. Algorithmen werden verwendet, um zum Beispiel die emotionale Befindlichkeit anhand eines Text einzuordnen, um Suizid-Gedanken frühzeitig zu erkennen \cite{stasytisIME}. Ein weiteres Anwendungsszenario sind Text-to-Speech (deutsche Übersetzung: Text-zu-Sprache) Funktionalitäten, bei denen gesprochene Worte des Nutzers durch Algorithmen in Text umgewandelt werden, sodass ein Nutzer die Nachricht nicht mehr aktiv eintippen muss. Eine wichtige Teilaufgabe ist die Recherche und das Versehen von datenschutzrechtlichen Aspekten, die bei der Datenspeicherung und der Implementierung von \textbf{Instant Messaging Systemen} bestehen.

\section{Aufgabenstellung}
\label{sec:Aufgabenstellung}

Die konkrete Aufgabe ist es, einen Chatserver auf Basis des \textbf{\ac{XMPP}} in Betrieb zu nehmen über den sich mehrere Chat-Clients authentifizieren und verschlüsselt Nachrichten austauschen können. Durch eine gründliche Recherche soll eruiert werden, welcher Chatserver sich hierfür eignet und warum dieser Chatserver für das Projekt verwendet wird. Der Nachrichtenaustausch soll Ende-zu-Ende verschlüsselt erfolgen. Eine wichtige Teilaufgabe ist, dass die Software Gruppenchats verschlüsselt unterstützt. Eine Zustellung der Nachrichten in Echtzeit soll implementiert werden. Benutzer sollen Nachrichten über eine Web-Oberfläche eingeben und empfangen können. Es soll ein Prototyp eines Machine Learning Algorithmus der Kategorie \textbf{\ac{NLP}} implementiert werden und in die Web-Oberlfäche integriert werden. Eine wichtige Aufgabe ist, dass das gesamte Chatsystem datenschutzfreundlich implementiert und programmiert wird.

\section{Ziele der Arbeit}
\label{sec:Ziele}

Die Ziele der Arbeit sind es, eine geeignete Netzwerkumgebung und einen lauffähigen Chatserver auf Basis von \ac{XMPP} in Betrieb zu nehmen. Es sollen tiefe Kenntnisse und Erfahrungen mit dem Umgang des \ac{XMPP} Protokolls gesammelt werden und sich mit dem Aufbau des Nachrichtenprotokolls auseinandergesetzt werden. Außerdem sollen sich mit der Arbeitsweise und den Grundlagen von \textbf{Instant Messaging Systemen} und der Nachrichtenübertragung in Echtzeit vertraut gemacht werden. Es soll ein funktionsfähiger Prototyp einer WebUI entstehen, der einen \ac{MUC} unterstützt und über den Chatserver Nachrichten verschlüsselt versendet und empfängt. Durch Recherche und praktische Entwicklungen soll das Fachwissen in der Programmiersprache \textbf{Python} vertieft werden. Zusätzlich sollen Fähigkeiten im Bereich \textbf{Machine Learning} erlernt werden und ein lauffähiger Prototyp eines \ac{NLP}-Models in die Weboberfläche integriert werden. Das letzte Ziel beinhaltet, Kenntnisse im Bereich des Datenschutzes bei Chatapplikationen zu erarbeiten und den Chatserver datenschutzfreundlich zu konfigurieren.

\section{Stand der Technik}
\label{sec:StandDerTechnik}

\textbf{Instant Messaging Systeme} existieren in der heutigen Form seit Ende der 1990er Jahre, ausgehend von einer Öffnung des Internets für einen größeren Nutzerkreis außerhalb von Forschungsinstitutionen. Das erste \ac{IMS}, welches eine größere Verbreitung fand, war \textbf{\ac{ICQ}} der Firma Mirabilis, welches Benutzern ermöglicht in einer grafischen Oberfläche Nachrichten untereinander oder in Chatrooms auszutauschen \cite{ICQ}. Die Systeme werden in unterschiedlichen Ausprägungen stets weiterentwickelt. Es kommen zur grundlegenden Funktion des Nachrichtenaustauschs weitere Features, wie Dateiübertragung, \textbf{\ac{VoIP}}, Video over IP, \textbf{Ende-zu-Ende-Verschlüsselung} oder Sprachnachrichten hinzu \cite{gross2007}. 

\newpage

\chapter{Datenschutzrechtliche Aspekte}
\label{chap:Datenschutz}

\newpage

\chapter{Anforderungen}
\label{Anforderungen}

\newpage

\chapter{Theoretische Grundlagen}
\label{chap:Theorie}

\section{XMPP-Protokoll}
\label{sec:XMPP}
XMPP stellt ein Protokoll auf Basis von XML dar. Das Ziel bei der Entwicklung von XMPP war zur Verwendung bei Instant Messaging. Xmpp als Protokoll kann sowohl zwischen den Servern, als auch inmitten von Servern fungieren. Dabei nutzt es das Internet und erlaubt den Usern Sofortnachrichten an andere Anwender innerhalb des Internets zu schicken. Vor allem bei Sofortnachrichten kommt der Vorteil der Echtzeit von XMPP zum Einsatz. Weiterer positiver Effekt ist die Unabhängigkeit von Betriebssystemen, da Xmpp ein Protokoll auf Internetbasis darstellt. Außerdem ist XMPP auch nahezu Browserunabhängig.
{\url{https://www.computerweekly.com/de/definition/XMPP-Extensible-Messaging-and-Presence-Protocol}} (Abgerufen:15.01.2020) 

XMPP wird als Extensible Messaging and Presence Protocol definiert. Übersetz man dies detailliert ins deutsche so ergibt sich ein erweiterbares Nachrichten- und Anwesenheitsprotokoll. Eine Definition die Xmpp sehr gut beschreibt. Außerdem basiert XMPP auf XML, welches eine Markup Sprache darstellt. Laut dem RFC6120 lässt sich mittels XMPP Daten zwischen zwei oder mehreren Netzwerkeinheiten nahezu in Echtzeit austauschen. XMPP lässt sich in viele verschiedene Funktionen aufteilen, weshalb der grundlegende Zweck von XMPP ein leichteres Ziel verfolgt. Die Idee von XMPP ist es den Austausch von kleinen Teilen strukturierter Daten (\glqq XML stanzas\grqq) zwischen einem oder mehreren Netzwerkteilnehmern zu ermöglichen. Primär wird XMPP mithilfe einer Client-Server-Architektur implementiert, bei der sich ein Client mit einem Server verbindet, um mit anderen Teilnehmern Daten auszutauschen. Wird diese Architektur implementiert, so ist in der Regel der Ablauf definiert durch:
\begin{enumerate}
	\item Bestimmen der IP-Adresse und des Ports zu dem sich verbunden werden soll 
	\item Eine TCP Verbindung öffnen/aufbauen
	\item Öffnen eines XML streams über TCP
	\item Zur Verschlüsselung verwenden von TLS
	\item Zur Authentifizierung verwenden des SASLs Frameworks
	\item Eine Ressource an den XML stream anbinden
	\item Austausch unbegrenzter \glqq XML stanzas\grqq ()=> kleine Teile strukturierter Daten) mit anderen Netzwerkteilnehmern
	\item Schließen des XML streams
	\item Schließen der TCP Verbindung
\end{enumerate} 
{\url{https://tools.ietf.org/html/rfc6120}} (Abgerufen:16.01.2020)

Für den Austausch von Daten gibt es zwei elementare Konzepte. Zum einen die XML streams und die XML stanzas. Diese zwei Konzepte werden definiert um das Verständnis des Datenaustausch , wie es im obigen Ablauf definiert ist, zu erlangen. Bei dem Austausch der Daten mittels XML streams wird von einem Container zwischen den Teilnehmern gesprochen. Der XML stream ist durch den \glqq stream header\grqq (z.B. XML <stream>) und dem Ende des stream, dargestellt durch XML </streams>, eindeutig definiert. Die Anzahl der austauschbaren XML Elemente ist unbegrenzt und durch die Lebensdauer des streams definiert. Das XMP stanza wird nun als diskrete semantische Einheit strukturierter Daten, das von einem Teilnehmer zu einem anderen über den XML stream gesendet wird bezeichnet. Das XML stanza ist eine direktes Kind-Element des streams. Ein stanza kann wiederum selbst Kind-Elemente enthalten, das den XML stream definiert. Im Kern fungiert der XML stream wie eine Hülle um alle XML stanzas, die während einer Session versendet werden. Ein Aufbau kann wie folgt repräsentiert werden.
{\url{https://tools.ietf.org/html/rfc6120#section-4.1}} (Abgerufen:16.01.2020)


\section{Python}
\label{sec:Python}

\section{Ejabberd}
\label{sec:ejabberd}
Ejabberd:
Ejabberd ist ein freier XMPP-Server. Vorteil ist, dass die Software unter Betriebssysteme wie Windows und Unix-ähnlichen lauffähig ist. Ejabberd ist eine Abkürzung und steht für \glqq Erlang Jabber Daemon\grqq. Wie die Definition zeigt bezieht sich ejabberd auf die Programmiersprache Erlang. Grund hierfür ist, dass die ejabberd Software in Erlang geschrieben ist.
\url{https://de.wikipedia.org/wiki/Ejabberd} (Aufgerufen: 16.01.2020)

Ejabberd ist einer der bekanntesten XMPP-Server auf der Welt und kann in vielerlei Hinsicht verwendet werden. Sowohl Großprojekte als auch kleine Instanzen machen sich die Vorteile von ejabberd zum Vorteil. Der Start von ejabberd ist dem Jahr 2002 zuzuordnen. Seit dem Start wurde es von Grund auf für die Unternehmensbereitstellung entwickelt, vor allem mit dem Ziel robust zu sein. Aufgrund davon, dass der Fokus, zum damaligen Zeitpunkt, auf dem unternehmerischen Zweig lag, war es wichtig die Fehleranfälligkeit von ejabberd zu minimieren. Ein Vorteil der sich bis zum heutigen Zeitpunkt bewahrt hat. Außerdem kann ejabberd die Ressourcen mehrerer geclusterter Systeme nutzen. Des Weiteren besitzt ejabberd die Eigenschaft der Skalierbarkeit, indem die Kapazitäten mit wenig Aufwand erhöht werden kann. Während der Entstehungsphase war das XMPP-Protokoll (welches in Kapitel … beschrieben wird) noch unter dem Namen Jabber bekannt. Ejabberd kann verschieden benutzt werden. Im Fall der Studienarbeit wir die Community Edition von ejabberd benutzt, welche als open source zur Verfügung steht. Neben der Community Edition gibt es auch noch Möglichkeiten einer Business Edition, welches vor allem für die großen Unternehmen mit besserem Support rund um die Uhr und größeren Funktionen konzeptioniert ist. Die Architektur eines ejabberd services erweitert die Kernfunktionen von XMPP, welches das Senden von Nachrichten ist, um Faktoren wie die Skalierbarkeit, Konfigurierbarkeit und Fehlertoleranz. Außerdem gilt die Architektur von ejabberd als Modular. Das bedeutet, dass es an den Zweck eines Projektes angepasst werden kann. Die Modularität bringt bspw. Funktionen wie Gruppenchat mit ein. Aufgrund der großen Anzahl an Modulen werden lediglich die Module, die für das Projekt relevant sind aufgelistet.
Relevante Funktionen:
\begin{itemize}
	\item Einzelchat
	\item Gruppenchat (auch als Multi-User Chat - MUC - bezeichnet)
	\item Offline Nachrichten 
	\item Web-Unterstützung
	\item Nachrichtenübermittlungsbestätigung
\end{itemize}
Eine wichtige Eigenschaft ist das Authentifizieren von Nutzern, welches ebenfalls von ejabberd unterstützt wird. Dafür kann ejabberd sowohl mit einer internen, als auch mit einer externen Datenbank zusammen arbeiten. Aufgrund der genannten Eigenschaften und Funktionen von ejabberd, werden eine Vielzahl an mögliche Anwendungsgebieten abgedeckt. Während für viele kleine Projekte die interne Datenbank Mnesia ausreichend ist, wird im Rahmen der Studienarbeit mit einer externen Datenbank gearbeitet.
\url{https://docs.ejabberd.im/get-started/}(Abgerufen:19.01.2020)

\section{Chatserver}
\label{sec:Chatserver}

\begin{lstlisting}[language=Python,caption=Example Listing Python,label={lst:Example}]
"""The first step is to create an SMTP object, each object is used for connection 
with one server."""

import smtplib
server = smtplib.SMTP('smtp.gmail.com', 587)

#Next, log in to the server
server.login("youremailusername", "password")

#Send the mail
msg = "
Hello!" # The /n separates the message from the headers
server.sendmail("you@gmail.com", "target@example.com", msg)
\end{lstlisting}


\section{Datenbank}
\label{sec:Datenbank}

\newpage

\chapter{Umsetzung und Implementierung}
\label{chap:Umsetzung}

\section{Vorbereitung}
\label{sec:Vorbereitung}

\section{Implementierung des Chatservers ejabberd}
\label{sec:ChatserverEntwicklung}

\subsection{Anbindung des Chatservers}
\label{subsec:Anbindung}

\subsection{Security}
\label{subsec:Security}

\subsection{Konfiguration von ejabberd}
\label{subsec:Konfiguration}

\newpage

\section{Entwicklung des Chat-Clients}
\label{sec:ClientEntwicklung}

\ac{HTML}

\subsection{Entwicklung der GUI}
\label{subsec:EntwicklungGUI}

\subsection{Implemintierung des Backends}
\label{subsec:Backend}

\newpage

\chapter{Datenanalyse}
\label{chap:Datenanalyse}

\newpage

\chapter{Ergebnis}
\label{chap:Ergebnis}

\newpage

\chapter{Fazit}
\label{chap:Fazit}


\end{onehalfspacing}
\newpage

\bibliography{Literatur}
\newpage
%setze Anhang mit appendix
%addpart, sodass Anhang im Inhaltsverzeichnis ohne Buchstaben auftaucht
\appendix
\addpart{Anhang}

\end{document}